The cosmic tidal reconstruction is a kind of quadratic statistics developed to extract large scale information from alignment of small scale structures.
It uses the anisotropic distortions on small scales powerspectrum to solve for the large scale tidal shear and hence gravitational potential.

Here, we present a complete 3 dimentional reconstruction algorithm that works best in close linear regions.

First, we smooth the density contrast to reduce the non-gaussianity.

(1) Convolve the field with a Gaussian kernal
$S(\bm{k})=e^{-k^2R^2/2}$, 
we take $R=1.25\ \mr{Mpc}/h$ \cite{2012:pen},
to reduce the complicated non-linear effects on small scales.

(2) Gaussianize the field, taking 
$\delta_g=\mathrm{ln}(1+\delta)$. 
This is to allieviate the problem that filter $W_i$ in Eq.(\ref{eq:wi}) heavily weights high density regions.


Second, we filter for the small scale structures that are most likely to be influenced by tidal force of large scale fields and calculate its variance. 
With it, we estimate the tidal force and reconstruct the large scale density field.

Consider only first order coupling between small and large scales,  
the distorted power spectrum \cite{2015:zhu} is given by
\begin{eqnarray}
\label{eq:powerdistort}
P(\bm{k},\tau)|_{t_{ij}}&=&P_{1s}(k,\tau)+
\hat{k}^i\hat{k}^jt_{ij}^{(0)}P_{1s}(k,\tau)f(k,\tau)
\end{eqnarray}
where $P(\bm{k},\tau$ can be obtained from observation, 
$f=2\alpha(\tau)-\beta(\tau)dlnP/dlnk$, 
$\alpha$ and $\beta$ are functions related to linear growth funcion \cite{2015:zhu}, 
and are calculated to be (0.6, 1.3) for $z=1$ and (0.4, 0.9) for $z=2$.
$P_{1s}(k,\tau)$ is the $1_{st}$ order small scale linear powerspectrum, 
from theoretical calculation.
(1) Following gravitational lensing procedures, decompose the symmetric, traceless tidal force tensor 

\begin{eqnarray}
\label{eq:tij}
t_{ij}=\Phi_{L,ij}-\nabla^2\Phi_L\delta^D_{ij}/3
\end{eqnarray}
into 5 components, 
\begin{eqnarray}
t_{ij}=\left( \begin{array}{ccc}
\gamma_{1}-\gamma_{z} & \gamma_{\times} & \gamma_{2}\\
\gamma_{\times} & -\gamma_{1}-\gamma_{z} & \gamma_{y}\\
\gamma_{2} & \gamma_{y} & 2\gamma_z
\end{array} \right).
\end{eqnarray}
Here, $\Phi_{L,ij}$ is the second derivative of large scale potential, 
$\delta^D$ is the Dirac function.


(2) Convolve $\delta_g$ with a filter $W_i$ 
deduced from Eq.(\ref{eq:powerdistort}) 
\begin{eqnarray}
\delta^{w_i}_g(\bm{k})=W_i(\bm{k})\delta_g(\bm{k}) 
\end{eqnarray}
Its effect is to select possible displacements caused by tidal field and calculate the variance.
\begin{eqnarray}
\label{eq:wi}
W_i(\bm{k})=i (\frac{P(k)f(k)}{P_{tot}^2(k)})^{\frac{1}{2}}\frac{k_i}{k}
=S(k)\frac{k_i}{k}\nonumber
\end{eqnarray}
where i indicates $\hat x,\hat y,\hat z$ directions, 
$P_{tot}=P+P_{noise}$ is observed matter powerspectrum, 
P is theoretical matter powerspectrum,
\footnote{The value of $S(k)$ on different scales could be seen in Appendix 1.}

(3) Estimate the 5 tidal tensor components from density variance.
\begin{eqnarray}
\label{eq:gamma}
\hat{\gamma}_1(\bm{x})&=&
[{\delta}^{w_1}_g(\bm{x}){\delta}^{w_1}_g(\bm{x})-
{\delta}^{w_2}_g(\bm{x}){\delta}^{w_2}_g(\bm{x})],\nonumber\\
\hat{\gamma}_2(\bm{x})&=&
[2{\delta}^{w_1}_g(\bm{x}){\delta}^{w_2}_g(\bm{x})],\nonumber\\
\hat{\gamma}_x(\bm{x})&=&
[2{\delta}^{w_1}_g(\bm{x}){\delta}^{w_3}_g(\bm{x})],\\
\hat{\gamma}_y(\bm{x})&=&
[2{\delta}^{w_2}_g(\bm{x}){\delta}^{w_3}_g(\bm{x})],\nonumber\\
\hat{\gamma}_z(\bm{x})&=&
[(2{\delta}^{w_3}_g(\bm{x}){\delta}^{w_3}_g(\bm{x})-
{\delta}^{w_1}_g(\bm{x}){\delta}^{w_1}_g(\bm{x})-
{\delta}^{w_2}_g(\bm{x}){\delta}^{w_2}_g(\bm{x}))]/3,\nonumber
\end{eqnarray}
(4) Reconstruct large scale density contrast $\kappa_\mr{3D}$ from tidal tensor:


With Eq.(\ref{eq:tij}) we get 
$\kappa_\mr{3D}\sim\nabla^2\Phi_L=\frac{3}{2}\frac{\partial_i\partial_j}{\nabla^2}t_{ij}$, hence
\begin{eqnarray}
\kappa_\mr{3D}(\bm{k})=\frac{1}{k^2}
&&[(k_1^2-k_2^2)\gamma_1(\bm{k})+2k_1k_2\gamma_2(\bm{k})\nonumber\\
&&+2k_1k_3\gamma_x(\bm{k})+2k_2k_3\gamma_y(\bm{k})\\\nonumber
&&+(2k_3^2-k_1^2-k_1^2)\gamma_z(\bm{k})].
\end{eqnarray}

Third, we correct bias and suppress noise with a Wiener filter.

Due to the foregrounds, the noise in z direction will be different from x,y direction, therefore we apply an anisotropic Wiener filter.
\begin{eqnarray}
	\label{eq:wiener}
\hat \kappa_{c}(\bm{k})=\frac{\kappa_{3D}(\bm{k})}{b(k_\perp,k_\parallel)}W(k_\perp,k_\parallel)\ ,
\end{eqnarray}
Bias $b=\frac{P_{k3D \delta}}{P_\delta}$, Wiener filter $W=\frac{P_\delta}{P_{k3D}/b^2}$.

Here $\hat \kappa_{c}$ is the output large scale density contrast we obtain from tidal reconstruction.
We use it to calculate velocity $\hat v_z^{tide}$ and mock kSZ signal $\hat \Theta_{tide}$ following identical procedure as to noise substracted field.

