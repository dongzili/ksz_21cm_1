In section \ref{ssec:tide} we present the result of the simplest tidal reconstruction. 
However, since our noises are strongly denpendent on direction and scale, 
there are several steps that can help obtain more accurate reconstruction.

First, there is a $P_{\mathrm{tot}}=P+P_\mathrm{noise}$ 
in the denominator of the filter $W_i$ in Eq.\ref{eq:wi}. 
Previously, we simply set $P_\mathrm{tot}=P$. 
A more accurate way to select relavant distortions is to consider different noise level of different scales. 
We can estimate the noise spectra in simulation with  
$P_\mathrm{noise}(k_\perp,k_\parallel)=P_{\delta_{ns}}(k_\perp,k_\parallel)-b^2(k_\perp,k_\parallel)P_\delta(k_\perp,k_\parallel)$, where b is again the bias. 
After that, we apply different renormalization to $\gamma$ in 
eq.\ref{eq:gamma}. 
The effect is to assign heavier weights to large $z$, where signals are cleaner.

Second, since different $\gamma$ use different modes, 
their noise and bias are different. 
It is better to filter them seperately before combining together. 
Follow Eq.\ref{eq:largepoten}, we could estimate the expected value of $\gamma$ 
with 
$t_{ij}\sim \frac{2}{3}\frac{k^2}{k_ik_j}\delta$, 
where $\delta$ is the original field with complete large scale structure. 
After that, we apply Wiener filter similar to Eq.\ref{eq:wiener} 
to each $\gamma$ before calculating $\kappa_\mathrm{3D}$. 
This measurement will better suppress the noises and assign heavier weights to 
shear estimators related to $z$ direction. 
This is again because of the more intact information of small scale structure in z direction.

