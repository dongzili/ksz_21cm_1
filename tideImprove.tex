Previously we go through standard tidal reconstruction procedure. 
However, since our noises are strongly denpendent on direction and scale, 
there are several steps that can help obtain more accurate reconstruction.

First, for Eq.\ref{eq:wi}, there is a $P_{\mathrm{tot}}=P+P_\mathrm{noise}$ 
in the denominator of the filter $W_i$. 
Previously, we simply use P instead of $P_\mathrm{tot}$. 
A more accurate way to select relavant distortions has to consider different noise level of different scales. 
We can estimate the noise spectra following similar procedure as Eq.\ref{eq:wiener}, 
$P_\mathrm{noise}(k_\perp,k_\parallel)=P_{\delta_ns}(k_\perp,k_\parallel)-b^2(k_\perp,k_\parallel)P_\delta(k_\perp,k_\parallel)$
After that, we apply different renormalization to $\gamma$s in 
eq.\ref{eq:gamma}. 
The effect is heavier weights assigned to large $z$, where cleanest signals come from.

Second, since different $\gamma$ use different modes, 
their noise and bias are different. 
It is better to filter them seperately before combining together. 
Follow Eq.\ref{eq:largepoten}, we could estimate the expected value of $\gamma$ 
with 
$t_{ij}\sim \delta\frac{2}{3}\frac{k^2}{k_ik_j}$, 
where $\delta$ is the original field which has complete large scale structure. 
After that, we apply Wiener filter similar to Eq.\ref{eq:wiener} 
to independent $\gamma$ before getting $\kappa$. 


