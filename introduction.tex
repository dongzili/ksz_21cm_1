While the baryon abundance of early universe is well fixed 
%by the cosmic microwave background (CMB),Big Bang Nucleosynthesis and Lyman-$\alpha$ forest 
\cite{Cooke14,Fukugita98,Komatsu11,Hinshaw13}, 
%a deficiency was noiticed in local universe.
for $z\lesssim 2$, large fractions of baryons are missing in observations.  
The majority of them are believed to reside in Warm-Hot Intergalactic Mediums (WHIM) with typical temperature of $10^5$ K to $10^7$ K \cite{Pen1999,Soltan06}, which is too hot and diffuse to detect.
Progress has been made in recent years to detect colder fraction of WHIM using absorption lines 
(eg, $\mathrm{H_I}$, broad $\mathrm{Ly\alpha, Mg_{II},Si_{II}, C_{II}, Si_{III}, C_{III}, Si_{IV}, O_{VI}, O_{VII}}$) \cite{Bregman07,Werk14}, 
yet not much detection can reach T$\gtrsim 10^6$ K. 
Besides, these detections are usually biased towards collapsed objects, 
 and metal lines have to suffer from the great uncertainty on ionization states 
and metalicity.
%However, the lines are usually limited to close circumgalactic medium, while at least 25\% of the baryons are bebieved to reside in more diffused region \cite{Dave10}. Moreover, the uncertainty in ionized fraction and metalicity is also a dormant volcan

A more promising tracer for missing baryons is the kinetic Sunyaev-Zel'dovich(kSZ) effect \cite{Sunyaev72,Sunyaev80,Vishniac87}, 
which results from Compton scattering between CMB photons with free electrons. 
The radial velocity of electrons will give photon a Doppler shift 
and hence leads to a 
secondary anisotropy in CMB temperature.
The kSZ signal has lots of advantages: 
First, it is contributed from the absolute majority of baryons, 
leaving alone only less than $10\%$ of baryons that 
reside in stars, remnants, atomic and molecular gases \cite{Fukugita04}. 
The fraction is rather stable.  
Second, it only relates to electron density and radial velocity, 
regardless the temperature, pressure or metalicity,  
%so it is easy to interprete.  
so no extra assumptions are needed to estimate baryon abundance.  
Third, velocity mainly results from large scale structure, 
therefore the method is less biased towards local mass contraction.  

Attractive as it is, 
due to the contamination from primary CMB, facility noises, 
thermal SZ effect and CMB lensing, 
it is difficult to filter for the kSZ signal independently. 
Worse still, the signal itself does not contain redshift information.
Therefore, previous approches tend to cross correlate it with 
galaxy spectroscopic surveys, 
which has large scale information and accurate redshift. 
Yet, it is difficult and costy for this kind of survey to cover large area 
and reach high redshift.  
%eg. using pairwise-momentum estimator \cite{Hand12} or velocity-field-reconstruction estimator \cite{Shao11,Li14}. 
%However since they all require spectroscopy of galaxies to provide accurate redshift, the sky volume and redshift range to apply the method is limited. 
A recent effort try to relax the condition using projected fields of galaxies 
from photometry surveys, which is instantly feasible \cite{Hill16}. 
However, projected fields only maintain the largest scale information in z direction, 
while for $l\gtrsim1000$, where primary CMB fades away, a sufficient amount of kSZ signal comes from intermediate scales. 
This limits the overall S/N it can reach.

In this paper we put forward a new tracer for cross correlation---$\mathrm{H_I}$ density field from 21cm intensity mapping. 
$\mathrm{H_I}$ 21cm spectra have accurate redshift information. 
%which enables us to reconstruct velocity field and 
%get better correlation with kSZ powerspectra. 
And intensity mapping, since it 
integrates signals 
rather than distinguishing individual galaxies, 
can accumulates contributions from weak sources 
and reach high S/N much faster. 
There are already several ongoing experiments aim at large sky coverage and claim to be able to reach $z\gtrsim1$ in very near future.
CHIME \cite{2014SPIE.9145E..22B}, Tianlai \cite{2015ApJ...798...40X}, 
HIRAX \cite{HIRAX} etc.
Therefore, this correlator is more feasible than large galaxy spectroscopic surveys, 
and more accurate that projected field.

However, the 21cm density field has its own drawback---the complicated foregrounds results from integration. 
While a cosmic signal in 21cm measurement is of the order of mK,  
foregrounds coming from Galactic emissions, telescope noise, 
extragalactic radio sources and Radio recombination lines, 
can reach the order of K \cite{DiMatteo04,Masui13}. 
Lots of techniques have been developed to filter the foregrounds, 
taking advantage of the attribute that they have fewer bright spectral
degrees of freedom \cite{Switzer15}.
Unfortunately, the manipulation will contaminates the smooth large scale structure in radial direction, 
and hence degrade the correlation with kSZ signal.
%Since the kSZ signal coming from a both density and velocity field, 
%and velocity is greatly related to large scale structures. 
%This drawback will inhibit the cross correlation behavior.

In this paper, 
%we first present the algorithm of how to construct velocity field 
%and cross correlate with kSZ signals. 
we first evaluate the influence of foregrounds and other noises 
on the cross correlation. 
We then apply a method, Cosmic Tidal Reconstruction \cite{2012:pen,2015:zhu}, 
to recover the large scale structure from its tidal influence on small scales 
and see the improvement on correlation coefficients. 

The paper is organized as follows: 
In section II, we demonstrate given a density field, how to correlate it with kSZ signal with velocity reconstruction, similar to \cite{Shao11}; 
In section III, we present the result of cross correlation with foreground filtered field 
and discuss the behavior; 
Then in section IV, we introduce the method of 3D tidal reconstruction, 
and present the correlation results after small k modes recovered, 
In section V, we discuss redshift space distortions and estimate statistical errors, 
%discuss redshift distortion and possible improvements; 
and we conclude at section VI.


Notes: 
Throughout the paper, We use the $z=1,2$ output of six $N$-body simulations from the
$\mr{CUBEP}^3\mr{M}$ code \cite{2013:code}, each evolving $1024^3$ particles in a $(1.2\mr{Gpc}/h)^3$ box. 
Simulation parameters are as follows: Hubble parameter $h=0.678$, baryon
density $\Omega_{b}=0.049$, dark matter density $\Omega_{c}=0.259$,
amplitude of primordial curvature power spectrum $A_s=2.139\times10^{-9}$ at 
$k_0=0.05\;\mr{Mpc}^{-1}$ and scalar spectral index $n_s=0.968$.

we use "$\wedge$" to denote recontructed fields as 
oppose to fields directly from simulations.
