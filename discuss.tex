Here is one of the first application of using 3 dimensional tidal reconstruction. 
When the method is first developed \cite{2012:pen}\cite{2015:zhu}, 
it only uses $\gamma_1$,$\gamma_2$ two shear estimator in x,y plane, 
concerning the redshift space distortion. 
However, in our case, the small scale structures in x,y direction are partly lost in foreground substraction, while large $k_z$ remains more intact, 
hence most effective parts of reconstruction are coming from the rest three $\gamma$ that have more contributions from z components. 
Therefore the 2D tidal reconstruction is definitely insufficient. 
As for the redshift space distortion, 
linearly it just induces an additional contraction in z direction 
$\delta^{rsd}(\bm{k})=(1+f\frac{k_z^2}{k^2})\delta(\bm{k})$. 
Therefore can be easily substracted by dividing the additional term 
before all the calculations. 
The non-linear effect will not matter much since the large k cutoff we apply will smear 
the small difference on small scales. 
(deletable: 
Actually, even if we do not substract redshift space distortion, 
since the foregrounds are also in z direction, 
the additional term assigns more weigh to large $k_z$ modes, 
where most clean signals come from; 
assigns more weigh to shear estimators related to covariance in z directions, 
where best recovered modes come from, 
and will results to a better reconstruction result
(probably discussed in detail in next paper).
)
Moreover, the inhancement induced by redshift space distortion in large $k_z$ will increase the S/N at that level, 
therefore improve the reconstruction results. 
In all, redshift space distortion is not a problem in this case, 
it is more like a blessing. 


In this paper, we discuss the possibility of cross correlating kSZ signal with 
21cm intensity mapping as a new probe to study baryon distributions.
{\it yet never mention baryon distribution....} 
We present the correlation results after foreground substraction and high k cut off from simulations at redshift 1 and 2. 
We recover large scale information lost in foregrounds with a 3D tidal reconstruction and obtain a $r>0.6$ correlation for $l\sim100-2000$, 
and $S/N>3$ for $l\sim 500-3000$ with Planck noise. 
This shows a promising future for this method.
 
