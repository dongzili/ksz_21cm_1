Since actual sky surveys can only resolve structures of certain scales, 
it is essential to understand which scales contribute most to the kSZ signals. 
To fulfill the purpose, we write Eq.(\ref{eq:ksz}) in Fourier space and do a qualitive analysis. 

The finite box size of $1200$ Mpc will only have obvious influence on modes with $k\lesssim0.005$, 
we can safely assume the integration on $z$ direction to be from minus infinity to plus infinity. 
Moreover, the term $a(z)H(z)f(z)$ in Eq.(\ref{eq:v}) does not vary much in one box, we assume it to be a constant for simplicity. 
Then the Fourier transformation is just the $k_z=0$ mode of the momentum 
$p_\parallel(\bm{k})$ in Fourier space. 
\begin{eqnarray}
    \label{eq:thetak}
    \Theta(\bm{l})\equiv&\Theta&({k}_x\chi,{k}_y\chi,0)\propto\int\, 
    d^3k^\prime\delta(\bm{l}/\chi-\bm{k}_\perp^\prime,k_\parallel^\prime) v_z(\bm{k^\prime})\nonumber\\
    %
    %
%    \xrightarrow[region]{linear}&\int& 
%d^3k^\prime\delta({\bm{k}}_\perp-\bm{k}_\perp^\prime,k_\parallel^\prime)\delta(\bm{k}^\prime)\frac{k_z^\prime}{k'^2}
    \end{eqnarray}
An essential feature of Eq.(\ref{eq:thetak}) is that
in transverse plane, density and velocity field of different scales are multiplied together;
 while in parallel direction, only $\delta$ and $v_z$ with identical $k_z$ will be coupled together.  

The contributions of $\delta(\bm{k})$ and $v_z(\bm{k})$ of different scales 
are implied in Fig.\ref{}. 
%and also implicit at Eq.(\ref{eq:v}), 
The velocity field contributes almost dominantly from large scales, 
leaving little contribution from $k_z>0.2 h/Mpc$ and $k_\perp> 0.02 h/Mpc$. 
%ignoring the vast k modes in this region. 
This makes it roughly an ellipitical selection functions in the convolution
---
It selects $\delta$ with similar range of $k_z$ yet has $k_\perp\sim\l/\chi-0.01$ h/Mpc.  
To make it clearer, let us assume that $|v(\bm{k})|$ drops fast enough for 
large k and could be demonstrated as a Delta Function centered at 
$(k_\perp,k_\parallel)=(0.01,0.1)$ h/Mpc. 
then it is immediately shown that 
$\Theta(\bm{l}) \int\, 
    d^3k^\prime\delta(\bm{l}/\chi-\bm{k}_\perp^\prime,k_\parallel^\prime) 
    \delta^D(0.01,0.1)
    \sim \delta(\bm{l}/\chi-0.01,0.1)$. 
Therefore, when generating kSZ signals, 
it is crucial to have large scale modes for $v_z$, 
yet which modes matter most for $\delta(\bm{k})$ 
depends on which $\ell$ we look at.  

To decide the targeted $\ell$ range, 
there are many factors to consider:  
the strength of the primary CMB, facility limits and 
other fluctuations on CMB such as thermal SZ effect\ref{} and CMB lensing\ref{}.
Consider using Plank\ref{15Plank} data 
in the 217 GHz band, where the tSZ signal vanishes, 
we demonstrate the angular powerspectrum of primary CMB, 
facility noises, 
and kSZ of redshift 1 and 2. 
As shown in Fig.\ref{fig:CMB}, 
only in the range of $\ell \sim 500-3000$, 
there is a chance for us to distinguish the kSZ signal.  

For this $\ell$ range, we demonstrate the essential modes 
for mock kSZ signal in both $v_z$ and $\delta$ fields with red ellipse 
in Fig.\ref{fig:k3v}. 

