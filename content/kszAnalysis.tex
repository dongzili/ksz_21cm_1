
To understand which $\bm{k}$ of $\delta$ are most necessary in reconstructing the kSZ signal, the first step is to clarify which angular scale of the kSZ we are interested in. 
As demonstrated in Fig.\ref{fig:cmb_21cm}, the kSZ effect is too faint to be distinguished until the primary CMB starts to fade away, at roughly $\ell>500$. 
It is possible to select a frequency band where the thermal SZ signal is negligible, then the dominant factor at high $\ell$ will be the CMB instrumental noise.  With existing Planck \cite{Planck2015} data at 217 GHz, $\ell \sim 500-3000$ will be the visible window for kSZ signal. The window could be extended to higher frequency with ACTpol and CMB-S4. \philnote{Which window? Because of what effects? Did you show this?}

The next step is to understand what role each scale plays in contributing to the kSZ signal at $\ell \sim 500-3000$. If we write Eq.(\ref{eq:ksz}) in Fourier space, and given that $g(\eta)$ varies slowly, we see that $\Theta(\bm{\ell})$ is propotional to the $k_z=0$ mode of the momentum field, as marked in Fig.\ref{fig:p},
%$p_\parallel(\bm{k})$. 
\begin{align}
    \label{eq:thetak}
    \Theta(\bm{\ell}) &\propto p_\parallel({k}_x\chi,{k}_y\chi,0)\\
     & \propto \int 
    d^3k^\prime\,\delta(\bm{\ell}/\chi-\bm{k}_\perp^\prime,k_\parallel^\prime) v_z(\bm{k^\prime})\nonumber.
    \end{align}
\philnote{eqnarray is depricated. Use the align environment instead.} The
convolution of $\delta$ and $v_z$ indicates the signal comes from the cross talk of $\bm{\ell}/\chi-\bm{k}_\perp^\prime$ and $\bm{k}_\perp$, with a sum over all $k^\prime$. 

Since $v_z \propto k_z/k^3$, its power drops fast at small scales. \philnote{compared to?} If we boldly analogize \philnote{approximate?} $v_z$ as a Dirac delta function $\delta^D(\bm{k}^\prime)$, $\Theta(\bm{\ell})$ will reduce to an integral over $\delta(\bm{\ell}/\chi,0) v_z(0,0)$, with integration over the other $k^\prime$ being negligible due to the faintness of $v_z(k^\prime\neq0)$. In reality, where $v_z(\bm{k})$ is not as sharp as a Dirac delta, the peak will be closer to ($k_\perp$,$k_\parallel$)=($0.01$,$0.1$) h/Mpc rather than (0,0). We then see  
that most of the kSZ signal is generated by the cross talk between the part of $v_z(\bm{k})$ with $k$ in a small ball surrounding (0.01,0.1) h/Mpc and the part of $\delta(k)$ with $k$ close to $\delta(\bm{\ell}/\chi,0.1)$ h/Mpc. This is demonstrated in Fig.\ref{fig:k3v}.
%Background color indicates the contribution of $|\delta(k)|$ and $|v_z(k)|$ 
%during Fourier transform. 
Comparing it with Fig.\ref{fig:cmb_21cm}, we notice that while the essential modes \philnote{what do you mean by the essential modes?} for $\delta$ are partly resolved, the large scale information dominating $v_z$ is almost completely filtered out of the 21 cm IM field. Therefore, to retrieve the cross-correlation between kSZ and 21cm IM fields we must first reconstruct the large-scale $v_z$ modes.
