In this paper, we have discussed a method for of cross-correlating the kSZ signal with 21 cm IM, a probe which is directly sensitive to the large-scale distribution of baryons. All of the calculations presented here are based on ongoing experiment conditions and realistic noise levels. 
The loss of large scale modes due to foreground cleaning is identified as the main challenge for obtaining reasonable cross correlation. 
To alleviate the problem, an algorithm that exploits the non-linear tidal coupling between scales 
is employed. 
It could retrieve $>70\%$ information of the large scale modes. 
With the reconstructed large scale modes, the velocity fields could be linearly calculated 
and coupled to density fields to generate a mock kSZ map. 
The tightness of the correlation between the mock kSZ map and real kSZ is directly related to the S/N we could obtain in 
real observations. 

With existing Planck data, it is reasonable to expect a S/N of at least $\sim15$ at redshift 2 with data from CHIME, and HIRAX will yield $~50$ S/N due to the increased small-scale resolution provided by longer baselines. 

The next step for achieving better correlation is to retrieve the low $k_z$, high $k_\perp$ modes in density fields which are also lost in foregrounds. 
These modes are supposed to couple with $v_z(\mr{low} k_z,\mr{low} k_\perp)$ and generate a great percentage of kSZ signals. 
Unfortunately, the high $k_\perp$ region is beyond the effective scale of tidal reconstruction. 
However, data from weak lensing and photo-z galaxy surveys, which contain only large-scale structure in the $z$ direction, may probably be use to compensate for the loss.
 
Cross-correlating the kSZ signal with 21 cm IM is promising due to its feasibility with near-term data and model independence. CHIME is about to start collecting data, and construction for HIRAX is under way, given the reported experiment timelines. It is reasonable to expect our method to be testable within the next five years. Moreover, the method does not rely on assumptions about velocity fields or the conditions of the interstellar medium, and therefore the results can be more easily understood. We expect our method to be useful tool for studying the baryon distribution up to redshift 2 or higher. Furthermore, the unique property of 21 cm IM of having both large sky coverage and accurate redshift information offers extra information into the diffuse baryonic structure at angular scales of $\ell\sim 1000-2000$, larger scales than probed by all the other similar methods proposed. This will foster understanding of stellar feedback at the scale of galaxy clusters and filaments and therefore the evolution of the large-scale structure.

