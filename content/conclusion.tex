In this paper, we discuss the possibility of cross correlating kSZ signal with 
21 cm intensity mapping to study baryon distributions. 
All the calculations are based on ongoing experiment condition and realistic noise scales. 
A holographic way of cross correlation is applied. 
Second order tidal coupling of different scales are employed to compensate for 
lost modes. 
With existing Planck data, 
it is reasonable to expect at least $~15$ S/N with data from CHIME, 
and more optimistic estimates will yield $~50$ S/N for redshift 2 with HIRAX. 
The main obstacle for optimal correlation 
is lack of low $k_z$ high $k_\perp$ data due to foregrounds. 
This leads to information waste in the reconstructed velocity field. 
However, data from weak lensing, photometric galaxy surveys, which 
contains only large scale structure in z direction, may 
compensate for that. 
 
This method is promising for its feasibility and model independence. 
CHIME has already started to collecting data, 
and HIRAX is also in a close flight. 
It is reasonable to expect it to be tested within five years. 
Moreover, the method does not rely on assumptions about velocity fields 
or interstellar medium conditions. 
Less misunderstanding will appear in interpreting results. 
It is reasonable to expect it to be a new 
reliable attemp to study baryon distributions up to 
redshift 2 or higher. 
This will foster the understanding of baryonic feedbacks of galaxies, 
and the condition in ISM.
