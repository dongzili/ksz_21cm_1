In this paper, we have discussed a method for of cross-correlating the kSZ signal with 21 cm intensity mapping, a probe which is directly sensitive to the large-scale distribution of baryons. All of the calculations presented here are based on ongoing experiment conditions and realistic noise levels. A holographic method for cross-correlation is applied. \philnote{This is the first time you ever mention this. What does this mean?} We have employed a method for reconstructing the large-scale modes, which typically will be lost to forgeground cleaning in realistic 21 cm IM surveys, which exploits the non-linear tidal coupling between scales. Our method, therefore, is essentially studying the four-point correlation function $<\delta\,\delta\,\delta\, T>$. This is easily understood since the local modulation of the power spectrum $<\delta_s \delta_s>$ is used to reconstruct large-scale modes $\delta_L$ and hence $v_z$, the two of which are then convolved $<\delta_s v_z> $ to mimic the kSZ signal. \philnote{Is the conclusion the best place to introduce this new discussion?}
%$<\delta_s \delta_s> \rightarrow \delta_L \rightarrow v_z$, 
%$<\delta_s v_z> \rightarrow $kSZ. 

With existing Planck data, it is reasonable to expect a S/N of at least $\sim15$ at redshift 2 with data from CHIME, and HIRAX will yield $~50$ S/N. \philnote{The improvement is due to mainly to the increased small-scale resolution provided by longer baselines?} The main obstacle for optimal correlation is the lack of low $k_z$, high $k_\perp$ data due to foregrounds. This leads to information waste in the reconstructed velocity field. \philnote{What information is wasted?} However, data from weak lensing, \philnote{or/and?} photo-z galaxy surveys, which contain only large-scale structure in the $z$ direction, could be use to compensate this effect.
 
This method presented here is promising for due to its feasibility with near-term data and model independence. CHIME has begun collecting data \philnote{not actually true, only the Pathfinder has collected data}, and construction for HIRAX is under way. \philnote{Given the reported experiment timelines?} It is reasonable to expect our method to be testable within the next five years. Moreover, the method does not rely on assumptions about velocity fields or the conditions of the interstellar medium, and therefore the results can be more easily understood. \philnote{In comparison to what?} We expect our method to be useful tool for studying the baryon distribution up to redshift 2 or higher. Furthermore, the unique property of 21 cm IM of having both large sky coverage and accurate redshift information offers extra information into the diffuse baryonic structure at angular scales of $\ell\sim 1000-2000$, larger scales than probed by all the other similar methods proposed. This will foster understanding of stellar feedback at the scale of galaxy clusters and filaments and therefore the evolution of the large-scale structure.
