Until now only linear theory has been considered in the reconstruction. However, to extract the lost large-scale information, we need to consider couplings between different scales which arise in non-linear perturbation theory. 
%While there are a great variety of nonlinear effects, identifying a single one will help supress the noise. 
Here, we present the algorithm we use to solve for the large-scale structure, based on tidal couplings \cite{2015:zhu,2012:pen}.  

The evolution of small-scale structure is modulated by large-scale tidal forces. Consider only the anisotropic influence from tidal force, the generated distortions on matter power spectrum will be 
\begin{align}
\label{eq:powerdistort}
\Delta P(\bm{k},\tau)|_{t_{ij}} &=
\hat{k}^i\hat{k}^jt_{ij}^{(0)}P_{1s}(k,\tau)f(k,\tau),
\end{align}
where $P_{1s}(k,\tau)$ is the linear power spectrum, $f(k)=2\alpha(\tau)-\beta(\tau)d\ln P/d\ln k$ is the tidal coupling function, with $\alpha$ and $\beta$ parameters related to the linear growth funcion \cite{2015:zhu}. For example, $(\alpha,\beta)=(0.6,1.3)$ and $(0.4, 0.9)$ for $z=1$ and $z=2$, respectively. 
$t_{ij}$ is the tidal force tensor, which is symmetric and traceless and hence can be decomposed into five independent observables:  
%following gravitational lensing procedures.
\begin{eqnarray}
t_{ij}=\left( \begin{array}{ccc}
\gamma_{1}-\gamma_{z} & \gamma_{2} & \gamma_{x}\\
\gamma_{2} & -\gamma_{1}-\gamma_{z} & \gamma_{y}\\
\gamma_{x} & \gamma_{y} & 2\gamma_z
\end{array} \right).
\end{eqnarray}
Therefore, from the spatial dependence of the distortions $\Delta P(\bm{k},\tau)$, we can solve for different components of $t_{ij}$. The tidal forces are related to second derivative of large scale gravitational potential $\Phi_L$,  
\begin{eqnarray}
\label{eq:tij}
t_{ij}=\partial_i\partial_j\Phi_{L}-\nabla^2\Phi_L\delta_{ij}/3 /,
\end{eqnarray}
where $\delta_{ij}$ is the Kronecker delta function.

Different components of $t_{ij}$ are related to different $k$ modes of the large-scale density contrast $\kappa_\mathrm{3D}$.
\begin{eqnarray}
    \label{eq:largepoten}
    \kappa_\mr{3D}=\nabla^2\Phi_L=\frac{3}{2}\nabla^{-2}\partial_i\partial_j t_{ij}
\end{eqnarray}
We note that $f(k,\tau)$ increases with $k$ at the scales we are interested in, indicating the distortions are more manifest on small scales. This reinforces the feasibility of using observable high $k$ modes of the 21 cm IM field to extract low $k$ modes. 
\if2=1
We apply identical filters as \ref{15Hongming} 
when selecting the distortions. 
However, \ref{15Hongming} only harness $t_{ij}$ components 
in transverse plane, i.e.$\gamma_1$ and $\gamma_2$ 
as to avoid bias from redshift distortions. 
We use all five components, since there are more intact modes in 
z direction than transverse plane, 
and $\gamma z$ is actually the most important 
shear estimater for the $k_z>k_\perp$ region 
we are most interested in.

Programming steps:\noindent

(1) Gaussianize the field, taking 
$\delta_g=\mathrm{ln}(1+\delta)$. 
This is to allieviate the problem that filter $W_i$ in Eq.(\ref{eq:wi}) heavily weights high density regions.

(2) Following gravitational lensing procedures, decompose the symmetric, traceless tidal force tensor into 5 components, 

(3) Select density distortions caused by tidal force, 
by convolving $\delta_g$ with a filter $W_i$ 
deduced from Eq.(\ref{eq:powerdistort}) 
\begin{eqnarray}
\delta^{w_i}_g(\bm{k})=W_i(\bm{k})\delta_g(\bm{k}) 
\end{eqnarray}
\begin{eqnarray}
\label{eq:wi}
W_i(\bm{k})=i \bigg[\frac{P(k)f(k)}{P_{tot}^2(k)}\bigg]^{\frac{1}{2}}\frac{k_i}{k}
=S(k)\frac{k_i}{k}\nonumber
\end{eqnarray}
$P_{tot}=P+P_{noise}$ is the observed matter powerspectrum, 
and P is theoretical matter powerspectrum,
%\footnote{The value of $S(k)$ on different scales could be seen in Appendix 1.}

(4) Estimate the 5 tidal tensor components from quadratic statistics.
\begin{eqnarray}
\label{eq:gamma}
\hat{\gamma}_1(\bm{x})&=&
[{\delta}^{w_1}_g(\bm{x}){\delta}^{w_1}_g(\bm{x})-
{\delta}^{w_2}_g(\bm{x}){\delta}^{w_2}_g(\bm{x})],\nonumber\\
\hat{\gamma}_2(\bm{x})&=&
[2{\delta}^{w_1}_g(\bm{x}){\delta}^{w_2}_g(\bm{x})],\nonumber\\
\hat{\gamma}_x(\bm{x})&=&
[2{\delta}^{w_1}_g(\bm{x}){\delta}^{w_3}_g(\bm{x})],\\
\hat{\gamma}_y(\bm{x})&=&
[2{\delta}^{w_2}_g(\bm{x}){\delta}^{w_3}_g(\bm{x})],\nonumber\\
\hat{\gamma}_z(\bm{x})&=&
\frac{1}{3}[(2{\delta}^{w_3}_g(\bm{x}){\delta}^{w_3}_g(\bm{x})\nonumber\\
&&-{\delta}^{w_1}_g(\bm{x}){\delta}^{w_1}_g(\bm{x})
-{\delta}^{w_2}_g(\bm{x}){\delta}^{w_2}_g(\bm{x}))],\nonumber
\end{eqnarray}

(5) Reconstruct large scale density contrast $\kappa_\mr{3D}$ from tidal tensor:
\begin{eqnarray}
\kappa_\mr{3D}(\bm{k})=\frac{1}{k^2}
&&[(k_1^2-k_2^2)\gamma_1(\bm{k})+2k_1k_2\gamma_2(\bm{k})\nonumber\\
&&+2k_1k_3\gamma_x(\bm{k})+2k_2k_3\gamma_y(\bm{k})\\\nonumber
&&+(2k_3^2-k_1^2-k_1^2)\gamma_z(\bm{k})].
\end{eqnarray}

(6) Correct bias and suppress noise with a Wiener filter.

Due to the foregrounds, the noise in $z$ direction will be different from $x$,$y$ direction, therefore we apply an anisotropic Wiener filter.
\begin{eqnarray}
	\label{eq:wiener}
    \hat \kappa_{c}(\bm{k})=\frac{\kappa_{\mathrm{3D}}(\bm{k})}{b(k_\perp,k_\parallel)}W(k_\perp,k_\parallel)\ ,
\end{eqnarray}
Bias $b(k_\perp,k_\parallel)=P_{\mathrm{k3D,}\delta}/P_\delta$ 
is the cross powerspectra between reconstructed field $\kappa\mathrm{3D}$ and original field $\delta$, 
Wiener filter $W(k_\perp,k_\parallel)=P_\delta/(P_{\mathrm{k3D}}/b^2)$.

Here $\hat \kappa_{c}$ is the output large scale density contrast we obtain from tidal reconstruction.
We use it to calculate velocity $\hat v_z^{\mathrm{tide}}$.
\fi
