\begin{abstract}
    The obvious defficiency of baryon contents in observations for $z\lesssim 2$ 
    and its close correlation with baryon distributions, galaxy feedbacks and 
    interstellar medium states stand in the way of understanding structure 
    formations. 
%To step aside from strong emission of galaxies and resolve 
 To study the distribution of diffusive 'missing baryons', 
a large scale oriented probe, 
the kinematic Sunyaev-Zel'dovich (kSZ) effect 
on cosmic microwave background, was proposed.
However, its faintness and lack of redshift require another 
signal to cross correlate with it. 
Previous proposals either require large sky galaxy spectroscopic surveys, 
or existing photemetric surveys combined with full ACTPol/CMB-S4 data to 
obtain persuasive results,
which is hard to achieve in next five years. 
In this paper, a new possibility of cross correlating 
    kSZ with HI density from 21cm Intensity Mapping surveys is discussed. 
    The high effeciency and low facility requirements of the surveys  
make it possible for several ongoing experiments, eg. CHIME, to achieve large sky coverage 
    and consistently measure $z\lesssim2$ sky in next few years. 
Assuming realistic facility conditions and noise scales, 
we find that after retrieving noise smeared 
    information on large scales with nonlinear tidal coupling between scales, 
    a minimum of $~15$ S/N for both redshift 1 and 2 could be reached with CHIME and Planck. 
The fast construction of interferometers with longer baselines, eg. HIRAX, 
  may foster the S/N to reach $~50$ for redshift 2 with noise level of Planck.
\end{abstract}


