\begin{abstract}
    The prominent defficiency of baryon contents in observations for $z\lesssim 2$ 
    and its close correlation with feedbacks and 
    intergalactic medium conditions stand in the way of understanding structure 
    evolution. 
 To study the distribution of diffusive 'missing baryons', 
a large scale oriented probe, 
the kinematic Sunyaev-Zel'dovich (kSZ) effect 
on cosmic microwave background, was proposed.
However, its faintness and lack of redshift require another 
signal to cross correlate with it. 
Previous proposals usually require galaxy spectroscopic surveys 
and new generation ground based CMB experiments to 
obtain convincing S/N.
This induces constraints on redshift depth and sky coverage, 
and limit the study for scales larger than $\ell<2000$. 
In this paper, a new possibility of cross correlating 
    kSZ with HI density from 21cm intensity mapping is discussed. 
	Ongoing intensity mapping experiments, eg. CHIME, making use of all weak sources, 
	are able to cover large sky area 
    and consistently measure $z\lesssim2.5$ sky in next few years. 
This enable us to study structure evolution of $l\sim 1000-2000$, which is the scale of clusters and filaments. 
Taking into account of constraints of existing facilities 
and foregrounds, 
we refine large scale informations with nonlinear tidal effects. 
A minimum of $~15$ S/N for both redshift 1 and 2 could be reached with CHIME + Planck. 
The fast construction of interferometers with longer baselines, eg. HIRAX, 
  may foster the S/N to reach $~50$ for redshift 2 with noise level of Planck.
\end{abstract}





