\begin{abstract}
    The obvious defficiency of baryon contents in observations for $z\lesssim 2$ 
    and its close correlation with baryon distributions, galaxy feedbacks and 
    interstellar medium states stand in the way of understanding structure 
    formations. 
For a comprehensive detection including diffusive baryons, 
a large scale oriented probe, 
the kinematic Sunyaev-Zel'dovich (kSZ) effect 
on cosmic microwave background (CMB), was proposed.
However, its faintness and lack of redshift require another 
signal to cross correlate with it. 
Previous proposals either require large sky galaxy spectroscopic surveys, 
or existing photemetric surveys combined with full ACTPol/CMB-S4 data to 
obtain persuasive results,
which is hard to achieve in next five years. 
In this paper, a new possibility of cross correlating 
    kSZ with HI density from 21cm Intensity Mapping surveys is discussed, 
    Due to the high effeciency and low facility requirements, 
there are already ongoing experiments like CHIME could 
satisfy our need. 
Assuming realistic facility conditions and noise scales, 
we find that after using nonlinear tidal coupling to retrieve 
lost information in large scales, a minimum of $~15$ S/N for both redshift 1 and 2 could be reached with CHIME. 
With the construction of interferometers with longer baselines such as HIRAX, 
    the S/N could reach $~50$ for redshift 2.
\end{abstract}


