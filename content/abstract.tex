\begin{abstract}
The prominent defficiency of baryons in observations at $z\lesssim 2$ and its poorly understood relationship with stellar feedback and the local conditions of the intergalactic medium stand in the way of understanding the evolution of structure in the universe. The kinematic Sunyaev-Zel'dovich (kSZ) distortion of the Cosmic Microwave Background (CMB) has been proposed as a method for measuring these diffuse ``missing baryons'' on large-scales. However, the faintness and lack of redshift information of the method limit its usefulness, suggesting cross-correlation with another probe. Previous proposals require the combination of spectroscopic galaxy surveys with new generation ground based CMB experiments to obtain convincing signal-to-noise (S/N). This limits the redshift depth and sky coverage they could reach. In this paper, the new possibility of cross-correlating kSZ with the neutral hydrogen (HI) density from 21 cm intensity mapping is discussed and tested with simulations. Ongoing experiments, such as CHIME, will, in next few years, measure the HI distribution at $z\lesssim2.5$ over large fractions of the sky. The greatest challenge for our method is the loss of large-scale 21 cm modes due to the combined effects of foreground filtering and spatial loss of interferometers related to shortest baselines. We alleviate this problem by restoring the large scale modes from its tidal influence on small scales. A minimum S/N of $~15$ for both redshifts 1 and 2 could be reached with a cross-correlation of CHIME and Planck. Interferometers with longer baselines, such as HIRAX, will be able to produce S/N reaching $\sim 30$ for redshift 1 and $\sim 50$ for redshift 2.
\end{abstract}





