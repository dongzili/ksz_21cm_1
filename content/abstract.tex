\begin{abstract}
The kinematic Sunyaev-Zel'dovich (kSZ) effect 
on cosmic microwave background (CMB), induced by radial momentum of hot electrons, 
is a powerful tracer to probe baryon distributions.  
However, the signal is weak and lack of redshift information, 
hence another survey with spectroscopic redshift 
is typically required. 
%which largely limits the sky area and depth to apply the method. 
This largely limits the sky area and depth to harness kSZ. 
Here, we propose a new source for cross correlation--- H~I density field from 21 cm intensity mapping.
21 cm spectra provide accurate redshift 
and intensity mappings integrate weak diffuse spectra,  
and thus can survey large sky area with great depth in much shorter time with low costs. 

One main concern of the method is that 
the complicate 21 cm foregrounds 
will contaminate radial large scale information, 
and reduce the correlation with kSZ. 
For redshift 1 and 2, 
we model the noise filtering in simulations, 
and find out that after velocity reconstructions, 
there is $\gtrsim0.7$ correlation with kSZ signal for $\ell\gtrsim 800$, and it drops for smaller $\ell$.  
To improve the correlation for smaller $\ell$, we recover large scale modes from their tidal influence on small scale structures (Cosmic Tidal Reconstruction). 
Successfully recover $>90\%$ information at $k\sim 0.01 h/Mpc$,  
we obtain a correlation $r\sim0.6-0.8$ for $\ell\sim100-2000$. 
The overall S/N for $\ell\sim 300-4000$ assuming Planck noise scale can reach 
45 at $z=1$, and 59 at $z=2$. 
Since the reconstructed field and foreground filtered field 
are superior in different modes, it is easy to combine them and improve S/N for $\ell\sim 1000$.
\end{abstract}


