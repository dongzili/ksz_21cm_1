\begin{abstract}
 The kinematic Sunyaev-Zel'dovich (kSZ) distortion of the Cosmic Microwave Background (CMB) has been proposed as a method for measuring diffuse baryon distribution on large-scales. Previous proposals of kSZ measurement with large sky spectroscopic galaxy surveys are unlikely to be achieved in the next three years. In this paper, we discuss the possibility of cross-correlating kSZ signal with the neutral hydrogen (HI) density from 21 cm intensity mapping. 
 This method could give quick access to data. For example, CHIME will start to collect data on early 2017 and measure the HI distribution for $z=0.8-2.5$ over large fractions of the sky in the following few years. There are great concerns about this probe due to the loss of large-scale modes in radial direction from foregrounds. Besides, the large-scale perpendicular modes are usually not well sampled by interferometers. It is believed that there should not be cross correlation between fields from intensity mapping and other large scale tracers. 
 In this paper, we first simulate 
 We alleviate this problem by restoring the large scale modes from its tidal influence on small scales. A minimum S/N of $~15$ for both redshifts 1 and 2 could be reached with a cross-correlation of CHIME and Planck. Interferometers with longer baselines, such as HIRAX, will be able to produce S/N reaching $\sim 30$ for redshift 1 and $\sim 50$ for redshift 2.
\end{abstract}





