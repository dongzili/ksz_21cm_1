\begin{abstract}
The prominent defficiency of baryons in observations at $z\lesssim 2$ (\philnote{relative to extrapolation from Big Bang Nucleosynthesis?}) and its poorly understood relationship with stellar feedback and the local conditions of the intergalactic medium stand in the way of understanding the evolution of structure in the universe. \philnote{Best to use active sentences.} The kinematic Sunyaev-Zel'dovich (kSZ) distortion of the Cosmic Microwave Background (CMB) has been proposed as a method for measuring these diffuse `missing baryons' directly, on large-scales. However, the faintness and lack of redshift information of the method limit its usefulness, suggesting cross-correlation with another probe. Previous proposals require the combination of spectroscopic galaxy surveys with new generation ground based CMB experiments to obtain convincing signal-to-noise (S/N). This induces constraints on redshift depth and sky coverage, and limit the study to scales larger than $\ell<2000$. \philnote{Do you need to discuss the exact scales in the abstract? It doesn't seem to help your argument. } In this paper, the new possibility of cross-correlating kSZ with the neutral hydrogen (HI) density from 21 cm intensity mapping is discussed. \philnote{Using what technique: theory, N-Body, ...?} Ongoing experiments, such as CHIME, will, in next few years, measure the HI distribution at $z\lesssim2.5$ over large fractions of the sky. These experiments will be sensitive to structure evolution at $\ell\sim 1000-2000$, the scale of clusters and filaments. \philnote{Again is this necessary in the abstract? Also $\ell$ of 1000 seems very high, do you mean $\ell$ of 100?} The greatest challenge for our method is the loss of large-scale 21 cm modes due to the combined effects of foreground filtering and spatial loss of interferometers \philnote{What does this mean? If you mean that telescopes have a finite field of view, and so do not measure the whole sky, then I would call it that: finite field of view. If you mean that interferometers are not sensitive to the small modes within their field of view, than my feeling is that this is not correct. Telescopes like CHIME should be sensitive to all modes up to their angular resolution. } We alleviate this problem by applying non-linear tidal reconstruction to recover the long-wavelength modes. A minimum S/N of $~15$ for both redshifts 1 and 2 could be reached with a cross-correlation of CHIME and Planck. Interferometers with longer baselines, such as HIRAX, will produce S/N reaching $\sim 50$ at redshift 2.
\end{abstract}





