In this section, we present a holographic method 
to cross correlate kSZ with a density field, 
following \cite{Shao11}. 

\label{sec:kszRecon}
The CMB temperature fluctuations caused by kSZ effect is:
\begin{eqnarray}
\label{eq:ksz}
\Theta_{kSZ}(\hat n)\equiv\frac{\Delta T_{kSZ}}{T_{\mr{CMB}}}
=-\frac{1}{c}\int d\eta  g(\eta)  \bm{p}_\parallel\ ,
\end{eqnarray}
where $\eta(z)$ is the comoving distance at redshift z, $g(\eta)=e^{-\tau} d\tau/d\eta$ is the visibility function, $\tau$ is the optical depth to Thomson scattering, $\bm{p}_\parallel=(1+\delta)\bm{v}_\parallel$, with $\delta$ the electron overdensity, $\parallel$ indicates direction parallel to line of sight. 
We assume that $g(\eta)$ doesn't change significally in one redshift bin, 
and integrate $\bm{p}_\parallel$ along radial axis to get $\hat \Theta_{kSZ}$

Due to the cancellation of positive and negative velocity, its direct cross correlation between kSZ signal will vanish.
To take advantage of the known redshift and better maintain the one to one multiplication between velocity field and density contrast,   
we generate a mock kSZ signal from calculation of linear peculiar velocity. 
%\cite{Shao11}.
%In this way, we can at most maximize the correlation.

Assume we have a density contrast field $\delta=(\rho-\bar{\rho})/\bar{\rho}$, where $\bar\rho$ is the average density of a certain redshift slice. 
%Detailed steps are as follows.

(1) Estimate the velocity field:

In linear region, the continuity equation goes like:
$\dot \delta+\nabla \cdot \bm{v}=0$, 
where $\bm{v}$ is the peculiar velocity and $\delta$ is the matter overdensity. 
%Therefore, we obtain an estimator of velocity distribution from the density contract $\delta$:
Therefore, linear velocity is estimated as: 
\begin{eqnarray}
	\label{eq:v}
\hat v_z(\bm{k})=i a H f\delta(\bm{k})\frac{k_z}{k^2}\,
\end{eqnarray}
where $f=d\mathrm{ln}D/d\mathrm{ln}a$, $D(a)$ is the linear growth function, 
$a$ is the scale factor, $H$ is the Hubble parameter.

$v_z \propto k_z/k^2$, indicating the most prominent signal comes from small k mode, which corresponds to large scale structure. 

(2) Filter the noise: 
The term $k_z/k^2$ in Eq.(\ref{eq:v}) will amplify noises in small k, 
which should be suppressed. 
\begin{eqnarray}
	\label{eq:wienerv}
\hat v_z^c(\bm{k})=\frac{\hat v_z(\bm{k})}{b(k_\perp,k_\parallel)}W(k_\perp,k_\parallel)\ ,
\end{eqnarray}
Bias $b=P_{\hat v_z,v_z}/P_{v_z}$, Wiener filter $W=P_{v_z}/(P_{\hat v_z}/b^2)$.

(3) Calculate 2D kSZ map following Eq.(\ref{eq:ksz}).

(4) Calculate correlation coefficients:

Quantify tightness of correlation with: 
\begin{eqnarray}
	r\equiv \frac{P_{recon,real}}{\sqrt{P_{recon}P_{real}}}\,
\end{eqnarray}
Where $\hat \Theta_{kSZ}$ is reconstructed signal 
and $\Theta_{kSZ}$ is original signal directly from simulations.


