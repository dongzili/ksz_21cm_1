Main drawbacks of ongoing 21cm Intensity Mapping 
experiments are demonstrated with simple filters. 

1. Small scale noises:

The finite spacial and velocity resolution of facilities 
prevent us from resolving infinitely small structures in both parallel and perpendicular directions. 
Therefore we import a cut off scale $k_{max}$ with a Heaviside Function 
$H(k_{max}-k)$, dropping all the modes with k larger than $k_{max}$. 
%We choose a realistic $k_{max}$ based on baselines of ongoing experiments 
%\cite{HIRAX,2014SPIE.9145E..22B,2014SPIE.9145E..4VN,2012IJMPS..12..256C,2015ApJ...798...40X}, 
%and an ideal $k_{max}$ based on noise status of Planck Satellite.

2. Foreground noises:

Foregrounds coming from Galactic emissions, telescope noise, 
extragalactic radio sources and Radio recombination lines, 
could be three orders brighter than actual signals\cite{DiMatteo04,Masui13}. 
The process of foreground removal, taking advantage of its low spectral
degrees of freedom \cite{Switzer15}, 
will inevitably contaminates the smooth large scale structure in radial direction.  
To imitate the loss, a high pass filter $W_{fs}(k_\parallel)=1-e^{-k_\parallel^2R_\parallel^2/2}$ is applied to density contraction. 
%We again test a realistic case \cite{2013ApJ...763L..20M,Switzer13}
%with 
%only half of the signal resolved at $k_\parallel=0.08$ Mpc/h and $0.15$ Mpc/h 
%for redshift 1 and 2 respectively. 
%And a theoretical case \cite{15Shaw} with half of the signal lost 
%at $k_\parallel=0.02$ Mpc/h and $0.04$ Mpc/h for redshift 1 and 2. 
%$R_\parallel=15\ \mr{Mpc}/h$ for $z=1$ and $R_\parallel=8\ \mr{Mpc}/h$ for $z=2$, which gives
%$W_{fs}=0.5$ at
%$k_\parallel=0.08\ \mr{Mpc}/h$ and $0.15\ \mr{Mpc}/h$ respectively. 
%And a good yet still realistic case \cite{15Shaw} with 
%$R_\parallel=60\ \mr{Mpc}/h$ for $z=1$ and $R_\parallel=32\ \mr{Mpc}/h$ for $z=2$, which gives
%$W_{fs}=0.5$ at
%$k_\parallel=0.02\ \mr{Mpc}/h$ and $0.04\ \mr{Mpc}/h$. 
%This is realistic according to the condition of current 21 cm observations 
%\cite{2013ApJ...763L..20M,Switzer13}.

3. Spatial loss of inteferometers:

Large scale structure in perpendicular plane, due to the smoothness, 
will appear sharp in the visibility function of interferometers 
after Fourier Transformation. 
Unfortunately that sharp peak could not be resolved by interferometers 
due to the incomplete sampling  
caused by the finite length of the shortest baselines. 
Therefore, structures with angular sizes greater than a threshold $l_{min}$ 
are assumed to be lost in our simulations. 
%We do not assume worse cases there, since baselines as short as 20m 
%are already able to filter most of the important modes for velocity reconstruction. 

In sum, the observed 21 cm density contrasts after all the loss will appear as 
\begin{eqnarray}
\label{eq:ns}
    \delta_{nf}(\bm{k})=\delta(\bm{k})H(k_{max}-k)W_{fs}(k_\parallel)H(l-l_{min}),
\end{eqnarray}

The chosen parameters and reasons are presented in Table. \ref{tab:para}.

The demonstration of a worst case for $z=1$ is shown in Fig.\ref{fig:}. 
As we could see, essential modes for density field are partly resolved. 
However, the reconstruction of velocity field will be a total failure 
due to the spatial loss of inteferometers. 

If we directly use this density contrast to generate mock kSZ signal, 
its correlation r (Eq.(\ref{eq:r})) with real kSZ will be at most 0.2.
Luckily, till now we only use the linear theory for reconstruction, 
while nonlinearly modes of different scales are coupled. 
If we could identify a relatively clean nonlinear effect in density field, 
we would be able to retrieve the information needed for velocity reconstruction. 
The effect we choose is the tidal influence of large scale structure on small 
scales, following \cite{2015:zhu,2012:pen}.  





%With the noise filtered density contrast $\delta_{nf}$, we follow the procedure described in
%section \ref{sec:kszRecon} to generate a mock kSZ signal $\hat \Theta_{nf}$  
%and calculate cross correlation $r_{\Theta\hat\Theta_{nf}}$.

\begin{table}
\begin{tabular}{|m{2cm}|m{1.5cm}|m{1.5cm}|m{1.5cm}|m{1.5cm}|}
    \hline
     & \multicolumn{2}{|c|}{z=1} &\multicolumn{2}{|c|}{z=2}\\
     \hline
     & high foreground &low foreground&high foreground& low foreground\\
     \cline{2-5}
     \footnote{Foreground: smear $k_z\lesssim 0.08,0.02,0.12,0.03$ h/Mpc respectively. Parameters based on \cite{2013ApJ...763L..20M,Switzer13,15Shaw}}
     $R_\parallel$ Mpc/h
      & 15 & 60 & 10 & 40 \\
     \hline
     & CHIME & HIRAX & CHIME &HIRAX\\
     \cline{2-5}
     \footnote{Small scale noises: based on CHIME\cite{2014CHIME} and HIRAX\cite{HIRAX} 
     with 100 m and 200 m longest baseline respectively.}
     $k_{max}$ h/Mpc 
     & 0.6 & 1.2 & 0.4 & 0.8 \\
     \hline
     \footnote{Spatial loss of inteferomenter: assuming shortest baseline of 20 m.}
     $\ell_{min}$
     & \multicolumn{4}{|c|}{300} \\
     \hline
\end{tabular}
     \caption{Parameter of different noise filtering.}
     \label{tab:para}
\end{table}
