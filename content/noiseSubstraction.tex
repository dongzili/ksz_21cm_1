
Ideally, this velocity reconstruction method should 
retrieve $>90\%$ kSZ signals \cite{Shao11}. 
However, realistic 21cm IM experiments 
can only detect density fluctuations at certain scales, 
as illustrated in Fig.\ref{fig:cmb_21cm}. 
Three main factors lead to the loss of modes. 

1. The spacial and velocity resolution of facilities. 

Unlike galaxy surveys, the resolution of IM is confined to 
the resolution of facilities. 
It decides the smallest scale to be observed, 
and the effect could roughly be resembled with a Heaviside Function 
$H(k_\mr{max}-k)$. 
%We choose a realistic $k_\mr{max}$ based on baselines of ongoing experiments 
%\cite{HIRAX,2014SPIE.9145E..22B,2014SPIE.9145E..4VN,2012IJMPS..12..256C,2015ApJ...798...40X}, 
%and an ideal $k_\mr{max}$ based on noise status of Planck Satellite.

2. Foreground noise level:

Foregrounds from Galactic emissions, 
extragalactic radio sources and Radio recombination lines, 
together with telescope noises 
could be three orders brighter than targeted signals\cite{DiMatteo04,Masui13}. 
The process of foreground removal, taking advantage of its low spectral
degrees of freedom \cite{Switzer15}, 
will inevitably contaminates the smooth large scale structure in radial direction.  
To imitate the loss, we apply a high pass filter $W(k_\parallel)=1-e^{-k_\parallel^2R_\parallel^2/2}$. 
%We again test a realistic case \cite{2013ApJ...763L..20M,Switzer13}
%with 
%only half of the signal resolved at $k_\parallel=0.08$ Mpc/h and $0.15$ Mpc/h 
%for redshift 1 and 2 respectively. 
%And a theoretical case \cite{15Shaw} with half of the signal lost 
%at $k_\parallel=0.02$ Mpc/h and $0.04$ Mpc/h for redshift 1 and 2. 
%$R_\parallel=15\ \mr{Mpc}/h$ for $z=1$ and $R_\parallel=8\ \mr{Mpc}/h$ for $z=2$, which gives
%$W_{fs}=0.5$ at
%$k_\parallel=0.08\ \mr{Mpc}/h$ and $0.15\ \mr{Mpc}/h$ respectively. 
%And a good yet still realistic case \cite{15Shaw} with 
%$R_\parallel=60\ \mr{Mpc}/h$ for $z=1$ and $R_\parallel=32\ \mr{Mpc}/h$ for $z=2$, which gives
%$W_{fs}=0.5$ at
%$k_\parallel=0.02\ \mr{Mpc}/h$ and $0.04\ \mr{Mpc}/h$. 
%This is realistic according to the condition of current 21 cm observations 
%\cite{2013ApJ...763L..20M,Switzer13}.

3. The shortest baseline of inteferometers:

Current 21cm IM experiments are all carried on inteferometers. 
To avoid disturbance, two beams of a interferometer 
cannot be placed infinitely close. 
The shortest baseline length decides the largest 
angular scale it could probe.  
Structures with angular scale greater than a threshold of 
$l_\mr{min}$ will be drained out 
in the visibility function 
when cross correlating signals received from different spots. 
We again use a Heaviside function to assemble the effect. 
%We do not assume worse cases there, since baselines as short as 20m 
%are already able to filter most of the important modes for velocity reconstruction. 

Therefore, a realistic 21 cm density contrasts will appear as 
\begin{eqnarray}
\label{eq:ns}
    \delta_\mr{IM}(\bm{k})=\delta(\bm{k})H(k_\mr{max}-k)W(k_\parallel)H(l-l_\mr{min}),
\end{eqnarray}

Table.\ref{tab:para} 
lists several representive values for different parameters 
based on previous observations and predictions. 
Fig.\ref{fig:cmb_21cm} 
is a demonstration of density contrasts corresponding to 
$R_\parallel=15$ Mpc/h, $k_\mr{max}=0.6$ h/Mpc 
at $z=1$. 
With this field, 
we construct a momentum field $p_\parallel$ following Eq.(\ref{eq:ksz}). 
As demonstrated in Fig.\ref{fig:p}, 
it does not cover the modes related to kSZ signals. 

Actually, directly using $\delta_\mr{IM}$ of any of the parameters 
to reconstruct kSZ signal will 
yields a correlation coeffcient $r<0.2$ 
with observable kSZ signals 

%With the noise filtered density contrast $\delta_{nf}$, we follow the procedure described in
%section \ref{sec:kszRecon} to generate a mock kSZ signal $\hat \Theta_{nf}$  
%and calculate cross correlation $r_{\Theta\hat\Theta_{nf}}$.
\begin{table}
\begin{tabular}{|m{2cm}|m{1.5cm}|m{1.5cm}|m{1.5cm}|m{1.5cm}|}
    \hline
     & \multicolumn{2}{|c|}{z=1} &\multicolumn{2}{|c|}{z=2}\\
     \hline
     & high foreground &low foreground&high foreground& low foreground\\
     \cline{2-5}
     $R_\parallel$ Mpc/h
     \footnote{Foreground: smear $k_z\lesssim 0.08,0.02,0.12,0.03$ h/Mpc respectively. Parameters based on \cite{2013ApJ...763L..20M,Switzer13,15Shaw}}
      & 15 & 60 & 10 & 40 \\
     \hline
     & CHIME & HIRAX & CHIME &HIRAX\\
     \cline{2-5}
     $k_\mr{max}$ h/Mpc 
     \footnote{Small scale noises: based on CHIME\cite{2014CHIME} and HIRAX\cite{HIRAX} 
     with 100 m and 200 m longest baseline respectively.}
     & 0.6 & 1.2 & 0.4 & 0.8 \\
     \hline
     $\ell_\mr{min}$
     \footnote{Spatial loss of inteferomenter: assuming shortest baseline of 20 m.}
     & \multicolumn{4}{|c|}{300} \\
     \hline
\end{tabular}
     \caption{Parameters related to resolvable modes in 21cm IM}
     \label{tab:para}
\end{table}
