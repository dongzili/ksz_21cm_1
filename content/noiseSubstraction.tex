Main modes loss of ongoing 21cm intensity mapping 
experiments are demonstrated in this chapter with simple filters. 

1. Small scale noises:

The finite spacial and velocity resolution of facilities 
import a cut off scale $k_{max}$ on both parallel and perpendicular 
directions, filtering out information of smaller scales. 
It is resembled with a Heaviside Function 
$H(k_{max}-k)$, dropping $k>k_{max}$. 
%We choose a realistic $k_{max}$ based on baselines of ongoing experiments 
%\cite{HIRAX,2014SPIE.9145E..22B,2014SPIE.9145E..4VN,2012IJMPS..12..256C,2015ApJ...798...40X}, 
%and an ideal $k_{max}$ based on noise status of Planck Satellite.

2. Foreground noises:

Foregrounds from Galactic emissions, telescope noises, 
extragalactic radio sources and Radio recombination lines, 
could be three orders brighter than actual signals\cite{DiMatteo04,Masui13}. 
The process of foreground removal, taking advantage of its low spectral
degrees of freedom \cite{Switzer15}, 
will inevitably contaminates the smooth large scale structure in radial direction.  
To imitate the loss, a high pass filter $W_{fs}(k_\parallel)=1-e^{-k_\parallel^2R_\parallel^2/2}$ is applied to density contraction. 
%We again test a realistic case \cite{2013ApJ...763L..20M,Switzer13}
%with 
%only half of the signal resolved at $k_\parallel=0.08$ Mpc/h and $0.15$ Mpc/h 
%for redshift 1 and 2 respectively. 
%And a theoretical case \cite{15Shaw} with half of the signal lost 
%at $k_\parallel=0.02$ Mpc/h and $0.04$ Mpc/h for redshift 1 and 2. 
%$R_\parallel=15\ \mr{Mpc}/h$ for $z=1$ and $R_\parallel=8\ \mr{Mpc}/h$ for $z=2$, which gives
%$W_{fs}=0.5$ at
%$k_\parallel=0.08\ \mr{Mpc}/h$ and $0.15\ \mr{Mpc}/h$ respectively. 
%And a good yet still realistic case \cite{15Shaw} with 
%$R_\parallel=60\ \mr{Mpc}/h$ for $z=1$ and $R_\parallel=32\ \mr{Mpc}/h$ for $z=2$, which gives
%$W_{fs}=0.5$ at
%$k_\parallel=0.02\ \mr{Mpc}/h$ and $0.04\ \mr{Mpc}/h$. 
%This is realistic according to the condition of current 21 cm observations 
%\cite{2013ApJ...763L..20M,Switzer13}.

3. Spatial loss of inteferometers:

Large scale structure in perpendicular plane, due to the smoothness, 
will appear sharp in the visibility function of interferometers 
after Fourier transformation. 
Unfortunately that sharp peak could not be resolved by interferometers 
due to the incomplete sampling  
caused by the finite length of the shortest baselines. 
Therefore, structures with angular sizes greater than a threshold $l_{min}$ 
will be lost. 
%We do not assume worse cases there, since baselines as short as 20m 
%are already able to filter most of the important modes for velocity reconstruction. 

In sum, the observed 21 cm density contrasts after all the loss will appear as 
\begin{eqnarray}
\label{eq:ns}
    \delta_{nf}(\bm{k})=\delta(\bm{k})H(k_{max}-k)W_{fs}(k_\parallel)H(l-l_{min}),
\end{eqnarray}

The chosen parameters and reasons are presented in Table. \ref{tab:para}.

Demonstration of a filtered density field corresponding to 
$R_\parallel=15$ Mpc/h, $k_{max}=0.6$ h/Mpc 
in $z=1$ is shown in Fig.\ref{fig:}. 
Essential modes for density field are partly left,  
while the modes for velocity field are completely gone 
due to the spatial loss of inteferometers. 
Directly using the filtered density contrast to 
generate mock kSZ signal will yield $r<0.2$ in Eq.(\ref{eq:r}) 

Luckily, till now only linear theories are considered in reconstruction, 
while there are couplings between different scales in nonlinear theories. 
Identifying a single nonlinear effect will help 
us retrieve important modes for velocity reconstruction. 
Here, we present an algorithm to employ 
distortions from second order tidal coupling 
to solve for large scale structures 
\cite{2015:zhu,2012:pen}.  





%With the noise filtered density contrast $\delta_{nf}$, we follow the procedure described in
%section \ref{sec:kszRecon} to generate a mock kSZ signal $\hat \Theta_{nf}$  
%and calculate cross correlation $r_{\Theta\hat\Theta_{nf}}$.

\begin{table}
\begin{tabular}{|m{2cm}|m{1.5cm}|m{1.5cm}|m{1.5cm}|m{1.5cm}|}
    \hline
     & \multicolumn{2}{|c|}{z=1} &\multicolumn{2}{|c|}{z=2}\\
     \hline
     & high foreground &low foreground&high foreground& low foreground\\
     \cline{2-5}
     \footnote{Foreground: smear $k_z\lesssim 0.08,0.02,0.12,0.03$ h/Mpc respectively. Parameters based on \cite{2013ApJ...763L..20M,Switzer13,15Shaw}}
     $R_\parallel$ Mpc/h
      & 15 & 60 & 10 & 40 \\
     \hline
     & CHIME & HIRAX & CHIME &HIRAX\\
     \cline{2-5}
     \footnote{Small scale noises: based on CHIME\cite{2014CHIME} and HIRAX\cite{HIRAX} 
     with 100 m and 200 m longest baseline respectively.}
     $k_{max}$ h/Mpc 
     & 0.6 & 1.2 & 0.4 & 0.8 \\
     \hline
     \footnote{Spatial loss of inteferomenter: assuming shortest baseline of 20 m.}
     $\ell_{min}$
     & \multicolumn{4}{|c|}{300} \\
     \hline
\end{tabular}
     \caption{Parameter of different noise filtering.}
     \label{tab:para}
\end{table}
