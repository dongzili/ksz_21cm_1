\label{sec:21cm}
Ideally, the velocity reconstruction method diskussed in the previous section should retrieve a $>90\%$ cross-correlation with the real kSZ signal \cite{Shao11}. However, realistic 21 cm IM experiments can only detect density fluctuations at certain scales, as illustrated in Fig.\ref{fig:cmb_21cm}. Three main factors lead to the loss of modes. 
\begin{enumerate}
\item The angular resolution of the telescope:
 
In comparison to galaxy surveys, IM surveys sacrifice angular resolution in exchange for sky coverage and survey speed. The effect of this finite angular resolution on S/N of the cross-correlation can be roughly taken in to account with a Heaviside Function in $k_\perp$ space, $H(k_{\perp\,\mr{max}}-k_\perp)$.
%We choose a realistic $k_\mr{max}$ based on baselines of ongoing experiments 
%\cite{HIRAX,2014SPIE.9145E..22B,2014SPIE.9145E..4VN,2012IJMPS..12..256C,2015ApJ...798...40X}, 
%and an ideal $k_\mr{max}$ based on noise status of Planck Satellite.

\item Foreground noise level:

Foregrounds from Galactic emission, extragalactic radio sources and Radio recombination lines, together with the noise of the telescope could be three orders brighter than the targeted signal \cite{DiMatteo04,Masui13}. The process of foreground removal, which takes advantage of the low spectral degrees of freedom of the forgegrounds \cite{Switzer15}, requires the removal of the corresponding large-scale structure in the radial direction, as well. To imitate the loss, we apply a high pass filter $W(k_\parallel)=1-e^{-k_\parallel^2R_\parallel^2/2}$.
%We again test a realistic case \cite{2013ApJ...763L..20M,Switzer13}
%with 
%only half of the signal resolved at $k_\parallel=0.08$ Mpc/h and $0.15$ Mpc/h 
%for redshift 1 and 2 respectively. 
%And a theoretical case \cite{15Shaw} with half of the signal lost 
%at $k_\parallel=0.02$ Mpc/h and $0.04$ Mpc/h for redshift 1 and 2. 
%$R_\parallel=15\ \mr{Mpc}/h$ for $z=1$ and $R_\parallel=8\ \mr{Mpc}/h$ for $z=2$, which gives
%$W_{fs}=0.5$ at
%$k_\parallel=0.08\ \mr{Mpc}/h$ and $0.15\ \mr{Mpc}/h$ respectively. 
%And a good yet still realistic case \cite{15Shaw} with 
%$R_\parallel=60\ \mr{Mpc}/h$ for $z=1$ and $R_\parallel=32\ \mr{Mpc}/h$ for $z=2$, which gives
%$W_{fs}=0.5$ at
%$k_\parallel=0.02\ \mr{Mpc}/h$ and $0.04\ \mr{Mpc}/h$. 
%This is realistic according to the condition of current 21 cm observations 
%\cite{2013ApJ...763L..20M,Switzer13}.

\item The shortest baseline of inteferometers:

Current 21cm IM experiments are all carried on inteferometers. 
For CHIME-like facilities, with multiple beams installed on one disk, 
the calibration for cross correlation between two beams of the same disk 
are complicated. 
Therefore, we only consider results from cross correlating signals from distinct disks. 
Then the minimum spacing between disks is the shortest baseline, which decides the largest angular scale it could probe.  We again use a Heaviside function 
$H(\ell-\ell_\mr{min})$ to mimick this effect. 
%We do not assume worse cases there, since baselines as short as 20m 
%are already able to filter most of the important modes for velocity reconstruction. 
\end{enumerate}
Therefore, a realistic 21 cm density contrast will appear as 
\begin{eqnarray}
\label{eq:ns}
    \delta_\mr{IM}(\bm{k})=\delta(\bm{k})H(k_{\perp\,\mr{max}}-k_\perp)W(k_\parallel)H(\ell-\ell_\mr{min}).
\end{eqnarray}
where $k_\parallel = \ell / \chi$. 

Table.\ref{tab:para} lists several representive values for the different parameters introduced above, based on previous observations and predictions. Fig. \ref{fig:cmb_21cm} is an illustration of the relevant scales for the various observables, corresponding to $R_\parallel=15$ Mpc/h, $k_{\perp\,\mr{max}}=0.6$ h/Mpc at $z=1$. With this 21 cm IM field, we construct a momentum field $p_\parallel$ following Eq.(\ref{eq:ksz}). As demonstrated in Fig.\ref{fig:p}, it does not cover the modes necessary for a cross-correlation with kSZ.

Actually, directly using $\delta_\mr{IM}$ with any combination of the parameters above to generate kSZ map will only yield a correlation coeffcient $r<0.2$ with observable kSZ signals. 
%With the noise filtered density contrast $\delta_{nf}$, we follow the procedure described in
%section \ref{sec:kszRecon} to generate a mock kSZ signal $\hat \Theta_{nf}$  
%and calculate cross correlation $r_{\Theta\hat\Theta_{nf}}$.
\begin{table}
\begin{tabular}{|m{2cm}|m{1.5cm}|m{1.5cm}|m{1.5cm}|m{1.5cm}|}
    \hline
     & \multicolumn{2}{|c|}{z=1} &\multicolumn{2}{|c|}{z=2}\\
     \hline
     & high foreground &low foreground&high foreground& low foreground\\
     \cline{2-5}
     $R_\parallel$ Mpc/h
     \footnote{Foreground: filter out $k_z\lesssim 0.08,0.02,0.12,0.03$ h/Mpc respectively. Parameters based on \cite{2013ApJ...763L..20M,Switzer13,15Shaw}}
      & 15 & 60 & 10 & 40 \\
     \hline
     & CHIME & HIRAX & CHIME &HIRAX\\
     \cline{2-5}
     $k_{\perp\,\mr{max}}$ h/Mpc 
     \footnote{Small scale noise: based on CHIME\cite{2014CHIME} and HIRAX\cite{HIRAX} 
     with 80 m and 200 m longest baseline respectively.}
     & 0.6 & 1.2 & 0.4 & 0.8 \\
     \hline
     $\ell_\mr{min}$
     \footnote{Spatial loss of inteferomenter: assuming shortest baseline of 20 m.}
     & \multicolumn{4}{|c|}{300} \\
     \hline
\end{tabular}
     \caption{Parameters related to resolvable modes in 21cm IM}
     \label{tab:para}
\end{table}
