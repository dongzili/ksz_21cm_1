\begin{abstract}
The kinetic Sunyaev-Zel'dovich(kSZ) effect 
on the Cosmic Microwave Background(CMB), induced by radial momentum of hot electrons, 
is believed to be a powerful probe for baryon distributions.  
However, since it is weak and lack of redshift information, 
another survey with spectroscopic redshift 
is typically required, 
%which largely limits the sky area and depth to apply the method. 
which largely limits the sky area and depth to harness kSZ. 
Here, we propose a new source for cross correlation--- HI density field from 21cm intensity mapping.
21cm spectra provides accurate redshift 
and since intensity mapping combines weak diffuse spectra rather than detecting individual sources, 
it can survey large sky area with great depth in much shorter time with low costs. 

The main concern of the new source 
comes from the complicate foregrounds of 21cm intensity mapping, 
which will seriously contaminate large scale information of radial direction.
We test the idea assuming realistic foregrounds, 
and find that the 














	We propose a new way to study baryon abundance and distribution at mid redshift. %how to vaguely describe redshift 1-2?%
    We cross correlate density field from HI 21cm intensity mapping with temperature anisotropy of Cosmic Microwave Background(CMB) caused by raidal motion of free electrons, i.e. kinetic Sunyaev-Zel'dovich(kSZ) effect. 
	We put forward a 3D cosmic tidal reconstruction to recover the modes lost in 21cm foregrounds, 
    which will effectively promote the apprearance of correlation signals.

	We verify the idea with simulation outputs on $z=1, z=2$, taking into account of foreground substractions, facility resolutions, and redshift distortions.
	We successfully recover $>90\%$ information of the 21cm density field at $k\sim 0.01 h/Mpc$ after tidal reconstruction. 
	We obtain a $r>0.6$ correlation with origin kSZ signal from $l\sim100-2000$.
    Assuming the noise level of Planck, 
%    there will be $\sim3-20 \sigma$ signal from $l\sim 300-4000$ for $z\sim1$, 
%   $\sim3-40 \sigma$ signal from $l\sim300-5000$ for $z\sim2$.
    the S/N will exceed 3 from $l\sim 500-3000$.

	This is a very promissing probe to study diffused matter distribution. 
    1. It is less biased towards local density contraction, thanks to the kinetic nature of kSZ signals;
	2. It has precise redshift information from 21cm spectrum; 
    3. It is rather feasible to get required data of higher redshifts with large sky coverage. 
    The 21cm intensity mapping survey has huge advantage on survey speed, 
    facility requirements and costs comparing to spectrocscopic galaxy survey.

    Consider the simulated S/N level, data requirements, current and upcoming 21cm intensity mapping facilities, we are optimistic about the new probe.
\end{abstract}


