In this paper, we discuss the possibility of cross correlating kSZ signal with 
21cm intensity mapping as a new probe to study baryon distributions. 
A tomographic way of calculating cross correlation with estimated velocity field is applied.  
Correlation results are presented for redshift 1 and 2, considering foreground noises, 
finite telescope resolution, and redshift space distortions. 
The latter two will not matter much. 
However, the foreground noise will smear the correlation on large scales while leaving sufficient correlation on smaller scales such as $\ell\sim 1000$. 
In order to study the large scale baryon distribution, we recover modes lost in foregrounds 
with a 3D tidal reconstruction and obtain a $r>0.6$ correlation for $\ell\sim 100-2000$. 
After the reconstruction, we will likely be able to distinguish cross correlation signals from $\ell\gtrsim 500$. 
Assuming Planch noise, the total S/N can reach 45 for $z=1$ and 59 for $z=2$. 
This shows a promising future for this method.
 
