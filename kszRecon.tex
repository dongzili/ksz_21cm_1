Given a density field, due to the cancellation of positive and negative velocity, its direct cross correlation between kSZ signal will vanish.
Here, we apply a method that first calculates the linear peculiar velocity from density field, and then generate a mock kSZ signal following the same equation as the real signals. 
\cite{Shao11}.
In this way, we can at most maximize the correlation.
%We constructing a mock 2D map of kSZ signal that can be direc real kSZ signal 

Assume we have a density contrast field $\delta=\frac{\rho-\bar{\rho}}{\bar{\rho}}$, where $\bar\rho$ is the average density of a certain redshift slice. 

Basic steps to get the cross correlation are as follows.

(1) Estimate the velocity field:

In linear region, the continuity equation goes like:
$\dot \delta+\nabla \cdot \bm{v}=0$, 
where $\bm{v}$ is the peculiar velocity and $\delta$ is the matter overdensity. 

Therefore, we obtain an estimator of velocity distribution from the density contract $\delta$:
\begin{eqnarray}
	\label{eq:v}
\hat v_z(\bm{k})=i a H f\delta(\bm{k})\frac{k_z}{k^2}\,
\end{eqnarray}
where $f=\frac{d\mathrm{ln}D}{d\mathrm{ln}a}$, D(a) is the linear growth function, 
a is the scale factor, H is the Hubble parameter.

$v_z \propto \frac{k_z}{k^2}$, indicating the most prominent signal comes from small k mode, which corresponds to large scale structure. 
This further verify our motivation for tidal reconstruction procedure.

(2) suppress the noise in velocity field with a Wiener filter. 
This is because the term $\frac{k_z}{k^2}$ in Eq.(\ref{eq:v}) will strongly amplify noises in small k modes. 
\begin{eqnarray}
	\label{eq:wienerv}
\hat v_z^c(\bm{k})=\frac{\hat v_z(\bm{k})}{b(k_\perp,k_\parallel)}W(k_\perp,k_\parallel)\ ,
\end{eqnarray}
Bias $b=\frac{P_{\hat v_z,v_z}}{P_{v_z}}$, Wiener filter $W=\frac{P_{v_z}}{P_{\hat v_z}/b^2}$.

(3) Calculate 2D kSZ map.

The CMB temperature fluctuations caused by kSZ effect is:
\begin{eqnarray}
\label{eq:ksz}
\Theta_{kSZ}(\hat n)\equiv\frac{\Delta T_{kSZ}}{T_{\mr{CMB}}}
=-\frac{1}{c}\int d\eta  g(\eta)  \bm{p}_\parallel\ ,
\end{eqnarray}
where $\eta(z)$ is the comoving distance at redshift z, $g(\eta)=e^{-\tau} d\tau/d\eta$ is the visibility function, $\tau$ is the optical depth to Thomson scattering, $\bm{p}_\parallel=(1+\delta)\bm{v}_\parallel$, with $\delta$ the electron overdensity. 
We assume that $g(\eta)$ doesn't change significally in one redshift bin, 
and integrate $\bm{p}_\parallel$ along radial axis to get $\hat \Theta_{kSZ}$

(4) Calculate correlation coefficients.

We compare reconstructed kSZ signals $\hat \Theta_{kSZ}$ with kSZ signals $\Theta_{kSZ}$ directly from simulations. 
To quantify the tightness of correlation, we employ a quantity r: 
\begin{eqnarray}
	r\equiv \frac{P_{recon,real}}{\sqrt{P_{recon}P_{real}}}\,
\end{eqnarray}


