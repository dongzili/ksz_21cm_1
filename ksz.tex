\documentclass[aps,prd,twocolumn,showpacs,superscriptaddress,groupedaddress,nofootinbib]{revtex4}  % for review and submission
\usepackage{graphicx}  % needed for figures
\usepackage{dcolumn}   % needed for some tables
\usepackage{bm}        % for math
\usepackage{amsmath,amssymb}   % for math
\usepackage{aas_macros}
\usepackage{multirow}
\usepackage{color}
\usepackage{verbatim}
\usepackage{times}
%what I add
\usepackage{tabu}
\usepackage{hyperref}
\usepackage{url}
\usepackage{microtype}
\usepackage[capitalise]{cleveref}
\usepackage[all]{hypcap}
\usepackage[toc,page]{appendix}
%
\newcommand{\mr}{\mathrm}
% avoids incorrect hyphenation, added Nov/08 by SSR
\hyphenation{ALPGEN}
\hyphenation{EVTGEN}
\hyphenation{PYTHIA}
\newcommand{\yr}{\ensuremath{\,{\rm yr}}}
\newcommand{\cm}{\ensuremath{\,{\rm cm}}}
\newcommand{\m}{\ensuremath{\,{\rm m}}}
\newcommand{\km}{\ensuremath{\,{\rm km}}}
\newcommand{\pc}{\ensuremath{\,{\rm pc}}}
\newcommand{\Mpc}{\ensuremath{\,{\rm Mpc}}}
\newcommand{\K}{\ensuremath{\, {\rm K}}}
\newcommand{\mK}{\ensuremath{\, {\rm mK}}}
\newcommand{\psr}{\ensuremath{\,{\rm sr}^{-1}}}
\newcommand{\Hz}{\ensuremath{\, {\rm Hz}}}
\newcommand{\kHz}{\ensuremath{\, {\rm kHz}}}
\newcommand{\MHz}{\ensuremath{\, {\rm MHz}}}
\newcommand{\erg}{\ensuremath{\,{\rm erg}}}
\newcommand{\eV}{\ensuremath{\,{\rm eV}}}
\newcommand{\keV}{\ensuremath{\,{\rm keV}}}
\newcommand{\Jy}{\ensuremath{\,{\rm Jy}}}
\newcommand{\Msun}{\ensuremath{{M_\sun}}}
\newcommand{\Lsun}{\ensuremath{{L_\sun}}}
\newcommand{\Zsun}{\ensuremath{{Z_\sun}}}
\newcommand{\tsec}{\ensuremath{{\rm s}}}
\newcommand{\nur}{\ensuremath{{\nu_R}}}
\newcommand{\nul}{\ensuremath{{\nu_L}}}
\newcommand{\los}{l.o.s.}
\newcommand{\tcb}{\textcolor{blue}}

\begin{document}
% The following information is for internal review, please remove them for submission
\widetext
% the following line is for submission, including submission to the arXiv!!
%\hspace{5.2in} \mbox{Fermilab-Pub-04/xxx-E}

\title{Cross Correlating 21 cm Intensity Mapping fields with Kinematic Sunyaev-Zel'dovich Effect: Probing Missing Baryons at $z\sim1-2$}
%\title{Cross Correlating Tidal Reconstructed 21 cm Signal with Kinematic Sunyaev-Zel'dovich Effect: A New Probe for Missing Baryons at $z\sim1-2$}
\input content/author.tex
\input content/abstract.tex
\pacs{}
\maketitle

%{\it Introduction.}---
\section{Introduction}
\input content/introduction.tex
%\section{Velocity Reconstruction and kSZ signals Cross Correlation}
\section{Algorithm: kSZ Cross Correlation}
\input content/kszRecon.tex
%\section{Analysis: Requirements for Cross Correlation}
\section{Condition: kSZ + 21cm intensity Mapping}
In this section, we discuss the origin of kSZ signal 
on different structure scales, 
and compare it with scales resolvable in 
ongoing 21cm Intensity Mapping experiments. 
The main purpose is to give an intuitive picture of the possibility and
 difficulty of the cross correlation.
\subsection{kSZ properties}
\input content/kszAnalysis.tex
\subsection{21 cm Intensity Mapping Properties} 
%\subsection{Noisy }
\input content/noiseSubstraction.tex
%\subsection{Cross Correlation with Noise Substracted Field}
%\input content/controlResult.tex
%\subsection{Explanation of Correlation Coefficient Fluctuations}
%\input content/controlExplanation.tex
\section{Algorithm: Cosmic Tidal Reconstruction}
%\subsection{3D Cosmic Tidal Reconstruction}
\input content/tide.tex
\section{Result: Cross Correlation}
\input content/tideResult.tex
%\subsection{Possible Improvements on Tidal Reconstruction}
%\input content/tideImprove.tex
%\section{Discussion and error estimates}
%\section{Redshift Space Distortions}
%\input content/rsd.tex
\section{Statistical error and s/n}
\input content/error.tex
\section{Conclusion}
\input content/conclusion.tex
\section{Acknowledge}
\input content/acknowledge.tex
%\appendix*
%\section{Complimentary Plots}
%\input appendix.tex
\bibliographystyle{apsrev}
\bibliography{ksz}
\end{document}
